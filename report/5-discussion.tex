\section{Discussion}

\subsection{Inter-site Data Consistency: OHP vs. CRAL}
The robustness of our photometric analysis is corroborated by the comparison of data acquired from two distinct observatories: OHP and CRAL (Lyon). As illustrated in Figure \ref{fig:HR}, the Color-Magnitude Diagrams (CMDs) generated from both datasets exhibit remarkable morphological consistency. Both diagrams clearly resolve the Main Sequence (MS) and the characteristic Turn-Off point at similar magnitudes ($g \approx 11-12$) and colors. The scatter observed in the Main Sequence, attributed to unresolved binaries and photometric errors at fainter magnitudes, appears consistent across both observing runs, validating the quality of the reduction pipeline and the calibration process against the Pan-STARRS reference catalog. Consequently, the physical parameters derived can be considered independent of site-specific instrumental systematics.

\subsection{Statistical Constraints via Monte Carlo Simulation}

To complement the visual isochrone fitting and reduce its intrinsic subjectivity, we
attempted a Markov Chain Monte Carlo (MCMC) exploration of the parameter space
($\tau$, $\mu$, $E(B-V)$). The motivation was not to obtain definitive statistical
uncertainties given the known limitations of our photometric calibration but rather to
probe the degeneracies between age, distance and reddening in a more systematic way.

As detailed in Sect 4.3, several MCMC configurations were tested. When treating the
extinction coefficients $(A_g, A_r, A_i)$ as independent parameters, the sampler failed
to converge and explored a broad, degenerate region of parameter space. Replacing these
coefficients with a single reddening parameter $E(B-V)$ improved numerical stability,
but the chains still did not converge toward a unique and fully physical solution.
Instead, they preferentially moved toward the “young and reddened’’ branch of the
parameter space—a consequence of the intrinsic age reddening distance degeneracy and
the dominant contribution of faint stars on the lower Main Sequence in the likelihood.

\subsection{Visual vs. Statistical Analysis: Addressing Degeneracies}
Comparing the MCMC results with our preliminary visual estimates reveals both agreements and disagreements that highlight the complexity of isochrone fitting.

\begin{itemize}
    \item \textbf{Age and Metallicity:} The MCMC-derived age of $\sim 65/79$ Myr, confirms the "young" classification suggested by the visual fit ($\sim 100$ Myr). The large uncertainty on the linear age (even the imposibility to determined precisly) is an expected consequence of the logarithmic sampling of isochrones and the lack of a well-populated Red Giant Branch, which makes the exact pinpointing of the age difficult based solely on the Main Sequence.
    
    \item \textbf{Reddening ($E(B-V)$):} The statistical reddening value ($\sim 0.71$) is in excellent agreement with our visual estimate of $0.70$. This confirms that the horizontal shift required to match the Main Sequence color is robust and physically significant, pointing to a high dust column density along the line of sight.
    
    \item \textbf{Distance Modulus ($\mu$):} A discrepancy is noted between the visual modulus ($\mu \approx 11.3$, $d \approx 1.8$ kpc) and the MCMC result ($\mu \approx 10.5$, $d \approx 1.2/1.3$ kpc). This shift of $\sim 0.5$ mag illustrates the classic \textbf{distance-reddening degeneracy}. The MCMC sampler favored a solution with slightly higher extinction ($0.85$ vs $0.7$) and a closer distance to optimize the fit of the lower Main Sequence. However, both distance estimates ($1.2/1.3 - 1.8$ kpc) place the cluster within the same Galactic spiral arm structure, maintaining the coherence of the astrophysical context.
\end{itemize}

In conclusion, while the visual fit provided a satisfactory first-order characterization, the MCMC analysis offers a more statistically grounded solution, particularly for the extinction parameter, however, with a huge needs to improve the approachs we used especially for the uncertainty. The combination of these methods robustly identifies NGC 7654 (M52) as a young ($\tau < 100$ Myr), heavily reddened open cluster located in the Perseus Arm of the Milky Way.

\subsection{Comparison with Literature}

To validate the physical parameters derived from our isochrone fitting and statistical exploration, we compared our results with established values from the literature, specifically the CCD photometry analysis by \cite{pandey2001ngc7654} and the astrometric characterization based on Gaia DR2 data by \cite{cantatgaudin2018gaia}.

\textbf{Distance and the Gaia Benchmark:}
The geometric distance is undoubtedly the most critical parameter to constrain, as it breaks the degeneracy with intrinsic luminosity. \cite{cantatgaudin2018gaia} derived a distance of $d = 1764$ pc ($\mu \approx 11.23$) for NGC 7654 using precise parallax measurements from the Gaia DR2 release.
Our visual fit, which yielded a distance modulus of $\mu \approx 11.3$ ($d \approx 1.8$ kpc), shows remarkable congruence with this astrometric benchmark. In contrast, the MCMC solution ($\mu \approx 10.52$, $d \approx 1.275$ kpc) significantly underestimates the distance. This comparison strongly suggests that the MCMC sampler, driven by the densest regions of the CMD (the faint Main Sequence), drifted towards a local minimum characterized by a smaller distance and higher extinction, illustrating the limitations of purely statistical fitting in the absence of strong priors on parallax.

\textbf{Age and Evolutionary Status:}
Regarding the cluster's age, our estimates ranging from $\tau \approx 75$ Myr (MCMC) to $100$ Myr (visual) place NGC 7654 firmly in the category of young open clusters. This is consistent with the findings of \cite{pandey2001ngc7654}, who reported an age range of $50-100$ Myr based on $UBVRI$ photometry. The presence of a well-defined Main Sequence Turn-Off in our data, compatible with the 100 Myr PARSEC isochrone, reinforces this consensus.

\textbf{Interstellar Reddening:}
Our derived color excess values ($E(B-V) \approx 0.71$) are stongly related to what have been reported in earlier studies (e.g., $E(B-V) \approx 0.64$ in \cite{pandey2001ngc7654}). However, \cite{pandey2001ngc7654} also highlighted the presence of significant non-uniform extinction across the face of the cluster, with values fluctuating by $\Delta E(B-V) \sim 0.20$. The residual scatter observed in our Main Sequence, which prompted the MCMC sampler to favor a higher extinction solution to fit the redder edge of the stellar locus, is likely a signature of this differential reddening. This confirms that NGC 7654 is embedded in a complex interstellar environment, typical of young clusters.

In summary, while the MCMC analysis highlighted the internal degeneracies of the dataset, our visual interpretation, when contextualised with the Gaia distance scale, provides a set of parameters ($d \approx 1.8$ kpc, $\tau \sim 100$ Myr) that our study is still consistent with the literature available for M52.