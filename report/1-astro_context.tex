\section{Astrophysical context}
% general astrophysical context / interest of studying this type of object / of using this observational approach
% briefly mention the theoretical basis of the study
% briefly state the content of the report


The study of open clusters is a fundamental tool for understanding and testing models of stellar evolution. Since their stars formed from the same molecular cloud, they have similar ages, distances, and chemical compositions. This homogeneity allows direct comparison of observations of these clusters with theoretical models. \\
The M52 cluster (NGC 7654), located in the constellation Cassiopeia, is a young cluster rich in stars of different types. The construction and analysis of the color-magnitude diagram (the observed equivalent of the HR diagram) allows isochrones, which are theoretical curves of all possible positions for stars of the same age but different masses, to be adjusted and three major parameters to be deduced: distance, age, and extinction. The turn-off point provides particularly good information about age of the cluster, while the adjustment of the main sequence leads to the distance modulus.\\
This report is divided into three parts. The first part details the reasons for selecting the open cluster M52, as well as a description of the observations, the conditions during acquisition, and the instrumental characteristics. The next part focuses on the different stages of image processing and calibration procedures, including instrumental reduction (bias, flats, darks), photometric measurement, and data transformation into standard magnitudes. Finally, the third and last part is devoted to modeling, where the theoretical framework and isochrones used are discussed, as well as the method chosen to adjust them to the data. The results obtained are then discussed.
