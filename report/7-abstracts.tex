% the French abstract must be a faithful translation of the English one
% the abstracts and the details of the authors contributions must fit in one page.

\vspace*{-2cm}
\begin{resumeen}
In this paper, we present a study of the open cluster M52 (NGC 7654) based on observations obtained using the IRiS telescope at the Observatoire de Haute-Provence (OHP). The images obtained in the g', r', and i' filters were reduced using standard calibration procedures, which included correction by the Bias, Dark, and Flat masters created just before, followed by star detection using the DAOstarFinder algorithm. The reference stars in the Pan-SARRS DR1 catalog allowed us to correctly calibrate our aperture photometry in order to obtain the corresponding magnitudes in the AB system.  The color-magnitude diagrams obtained for the data collected at the OHP and in Lyon (CRAL) were compared to the Padova isochrones, and the adjustment of the latter highlights a well-defined main sequence as well as a turning point that is consistent with a population of young stars of approximately 100 Myr. The distance modulus $\mu = 11.3$ places M52 at about 1.8 kpc, and the color excess $E(B-V) = 0.7$ confirms the strong extinction along the line of sight, which is located in the galactic plane. The residual dispersion we observe around the main sequence can be explained by photometric uncertainties, the presence of unresolved binary systems, or possible differential reddening.
 
\end{resumeen}

\vskip5.em

\begin{resumefr}
  Nous présentons dans ce papier une étude de l'amas ouvert M52 (NGC 7654) à partir des observations qui ont pu être obtenues à partir du télescope IRiS de l'Observatoire de Haute-Provence (OHP). Les images obtenues dans les filtres g', r' et i' ont été réduites en utilisant les procédures standard de calibration, ce qui inclut la correction par les masters bias, dark et flat créés juste avant, suivie par une détection des étoiles à l'aide de l'algorithme DAOStarFinder. Les étoiles de référence du catalogue Pan-STARRS DR1 nous ont permis de calibrer correctement notre photométrie d'ouverture afin d'obtenir les magnitudes correspondantes dans le système AB. Les diagrammes couleur–magnitude obtenus pour les données acquises à l'OHP et à Lyon (CRAL) ont été comparés aux isochrones Padova, et l'ajustement de ces derniers met en évidence une séquence principale bien définie ainsi qu'un point de rebroussement qui est cohérent avec une population d'étoiles jeunes d'à peu près 100 Myr. Le module de distance $\mu = 11.3$ situe M52 à environ 1.8 kpc, et l'excès de couleur $E(B-V) = 0.7$ confirme la forte extinction sur la ligne de visée située dans le plan galactique. La dispersion résiduelle que nous observons autour de la séquence principale peut s'expliquer par les incertitudes photométriques, la présence de systèmes binaires non résolus ou encore par un possible rougissement différentiel.
\end{resumefr}

\vskip5.em

\begin{contribution}
The workload was distributed according to specific tasks. Data reduction was a fully collaborative effort, shared equally among Tom, Romain, and Shareen. For the CMD (Color-Magnitude Diagram) analysis, Romain took the lead, contributing approximately half of the work, with significant support from Tom. Shareen was the primary contributor to the Report, handling the majority of the writing and formatting. Finally, the MCMC (Markov Chain Monte Carlo) analysis was conducted entirely by Tom.
\end{contribution}

\vskip5.em

\begin{remerciements}
We would like to express our gratitude for the hospitality we received at the Haute-Provence Observatory and for the excellent working conditions that were made available to us. We would also like to thank our supervisors Bertrand Plez, Julien Morin, and Jean-François Gonzalez, who were present and attentive throughout the duration of this project. Their valuable advice, availability, kindness, and expertise enabled us to successfully complete this project. We would also like to sincerely thank Nicolas Bouché for the data he collected for us at CRAL, which enabled us to achieve better results.
\end{remerciements}

